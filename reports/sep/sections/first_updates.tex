\section{First updates} \label{ch:first_updates}

The discussed prototype is succesful in what its aim was, but it became apparent that some improvements can be done.

The prototype applies itself onto each instance of \name{mec-2}.
This is not optimal, because this approach makes it hard for other similar projects to implement into the \code{Mec2Element} without side effects.

It would be better if just one \name{deepmech} property would be added to the \code{Mec2Element} class, providing the necessary functionality and giving more power to the user whether to activate deepmech, or not.

To implement this approach, the prototype has to be rewritten.

Hitherto the \code{deepmech} object was applied to each \code{mec-2} element.
This should also be changed to make singular detectors interchangable.
This modularized approach aims to ease maintenance and make it easier to make improvements.

\subsection{Dividing frontend from backend}

The first step is to divide the things the user interacts with with the things that the user expects to happen.

\subsection{Make models interchangable}

\subsection{Implement into \name{mec2}}

\subsection{Improve \name{mec2} HTML element}