\section{Conclusion}

The process of handling constraints and their respective detection using the proposed approach needs a lot more processing than the detection of symbols.
Where symbols are still limited to be 32x32 pixel images crops be of any size.
Furthermore constraints are not just a classification problem, but the direction of the constraint has to be identified, too.

For inference other approaches exist; e.g. \name{YOLO}~\cite{Bochkovskiy2020}.
For the scope of this project this would go to far.

The implementation of \name{mec2} into an application which is able to run on various platforms that can render web content and can be extended is investigated in this project.
The system agnostic frontend enables developers to implement backends for the various different needs they may have.
In this project the implementation into desktop frameworks via \name{WPF} and \name{WinUI} are examined.
The possibility to outsource the inference to a webserver is also introduced, providing yet another way of distributing the wordloads.

Table~\ref{tab:benchmarks} shows that delegating the inference to a different backend may be worth it, considering other tasks as video processing may be desired in the future.

\begin{table}
    \label{tab:benchmarks}
    \caption{
        Time before and after calling the \code{predict} action in \code{DeepmechSlice}.
        Two predictions are made, because some implementations need extra time for initialization; e.g. JavaScript implementation compiles the code just in time, increasing performance in subsequent runs.}
\begin{tabular}{lrrr}
    \toprule
    Implementation: & tf.js (PWA) & frugally-deep (WPF) & tf (Python-Webserver) \\
    \midrule
    First prediction: & \(1895\)ms & \(18\)ms & \(53\)ms \\
    \midrule
    Second prediction: & \(345\)ms & \(18\)ms & \(22\)ms \\
    \bottomrule
\end{tabular}
\end{table}

The overhead created by implementations into other environments is little, if the PWA can be implemented using web-view components.
These are available in most modern frameworks.
This also makes it possible to bring this functionality to mobile applications; e.g.\ using the Qt-Framework of Windows Xamarin.

The flexibility of the PWA enables to include various inference models into various environments with very little overhead, making this approach very promising.
The PWA can be accessed via \url{https://deepmech.klawr.de}.

