\section{Conclusion}

The process of predicting constraints using the proposed approach needs a lot more processing than the detection of symbols.
Where symbols are still limited to be 32x32 sized images the span of crops can not be limited without losing their intended use.
Furthermore constraints are not just a classification problem, but the direction of the constraint has to be identified, too, which doubles the effort.

For inference other approaches exist; e.g. \name{YOLO}~\cite{Bochkovskiy2020}, but their exploration would be outside of the scope of this project.

The implementation of \name{mec2} into an application which is able to run on various platforms that can render web content and can be extended was investigated thoroughly during this project.
The system agnostic frontend using the \code{mec2} HTML element extended by \name{React} are flexible enough to implement backends fitting for various needs different platforms may have.
In this project the implementation into desktop frameworks via \name{WPF} and \name{WinUI} are examined.
The possibility to outsource the inference to a webserver is also introduced, providing yet another way of distributing the wordloads.
Table~\ref{tab:benchmarks} shows that delegating the inference to a different backend seems to be worth it, considering other tasks as video processing may be desired in the future, making performance a bigger issue.

\begin{table}
    \label{tab:benchmarks}
    \caption{
        Time before and after calling the \code{predict} action in \code{DeepmechSlice}.
        Two predictions are made, because some implementations need extra time for initialization; e.g. JavaScript implementation compiles the code just in time, increasing performance in subsequent runs.}
\begin{tabular}{lrrr}
    \toprule
    Implementation: & tf.js (PWA) & frugally-deep (WPF) & tf (Python-Webserver) \\
    \midrule
    First prediction: & \(1895\)ms & \(18\)ms & \(53\)ms \\
    \midrule
    Second prediction: & \(345\)ms & \(18\)ms & \(22\)ms \\
    \bottomrule
\end{tabular}
\end{table}

The overhead created by implementations into other environments is little, if the PWA can be implemented using web-view components.
These are available in most modern frameworks.
This also makes it possible to bring this functionality to mobile applications; e.g.\ using the Qt-Framework of Windows Xamarin, but as this project focusses on desktop applications this is not examined further.

The results of this case study are promising, as extending the PWA itself and implementing it seems to be feasible without an overhead to big.
The PWA can be accessed via \url{https://deepmech.klawr.de}, the other implementations are open source and licensed unter the MIT license, so they can be built without a permission.

