% !TeX encoding=utf8
% !TeX program = pdflatex
% !BIB = biber

\listfiles % listet alle geladenen Pakete im .log file
% sollte ihr .tech-file nicht kompilieren, so vergleichen Sie bitte Ihre Liste
% mit PaketListe.txt

\documentclass[paper=a5,
      fontsize=10pt,
      parskip=half-, %
      ngerman   % Neue Rechtschreibung, d.\,h. (Silbentrennung)
]{scrartcl}

\usepackage{packages} % Präambel laden
% NOTE selbstgemacht...
\newcommand\name[1]{\texttt{#1}}
\usepackage{float}
\usepackage{subcaption}
% NOTE selbstgemacht ende

% Laden der .bib Datei
\addbibresource{document.bib}

% TITEL des Beitrags % \LARGE zur Erzeugung korrekter Überschriftengröße
\title{\LARGE{Automatische Klassifizierung handgezeichneter Mechanismen durch maschinelles Lernen
}}

% AUTOREN (Vorname, Nachname)
\author{Stefan Gössner*, Kai Lawrence**}

% DATUM ausblenden
\date{}

% dieses Feld wird benutzt um die Kontaktadresse darzustellen
% bitte nutzen sie ein hochgestellten Stern (*) um bei mehreren Instituten zu
%unterscheiden
\publishers{* FH Dortmund, Professur für Dynamik, Mechnanismentechnik und Webtechnologien\\
	stefan.gössner@fh-dortmund.de
\\ \vspace{\baselineskip}
% Bei mehreren Instituten nutzen Sie bitte mehrere hochgestellte Sterne, um die
%Autoren zuzuordnen
** FH Dortmund, Maschinenbaustudent\\
	kaihenning.lawrence002@stud.fh-dortmund.de}


\begin{document}

% Wahl der Sprache des Dokuments (ngerman oder englisch)
\selectlanguage{ngerman}

\maketitle % erzeuge Titel

\section*{Kurzfassung}
Anwendungen für Mechanismentechnik leiden immer unter dem Problem, dass für eine schnelle Skizze eines Mechanismus die entsprechenden Benutzeroberflächen der bestehenden Mechanismeneditoren mit Aufwand verbunden sind. Es gibt zwar Anwendungen zur schnellen und intuitiven Erstellung von Mechanismen, aber für kurze Betrachtungen ist der Einsatz eines solchen Editors in vielen Fällen nicht der Mühe wert.

Um auch ad-hoc eine Möglichkeit zu haben Mechanismen interaktiv oder animiert zu betrachten wird nun ein Programm entwickelt, welches es ermöglichen soll Handzeichnungen zu scannen und den respektiven Mechanismus digital zu ermitteln.

\section*{Abstract}
Applications for mechanism engineering always suffer from the problem that for a quick sketch of a mechanism the corresponding user interfaces of existing mechanism editors are connected with effort.
Applications for a quick and intuitive sketching of mechanisms exist, but for short considerations using a computer is in many cases not worth the effort.

In order to be able to view mechanisms interactively or animatedly on an ad-hoc basis, a program is now being developed which will allow hand drawings to be scanned and the respective mechanism to be determined digitally.


\section{Introduction}

This project builds upon work previously done to learn the fundamentals of machine learning, which can be found at \url{aka.klawr.de/srp}.
The target was to introduce machine learning in to the field of kinematics mainly undertaken in web environment by creating a statistical model to classify handwritten mechanical symbols.

In this project this idea is taken further to detect whole mechanisms by determining the position of the symbols in unspecialized pictures\footnote{The data used to train the symbol classifier assumes symbols which are centered and fill the whole picture}.

After localization is done, the next problem is to detect the connections between nodes.
For this the images between nodes have to be cropped and processed to be able to get a reasonable prediction.
A new inference model is trained to classify these processed images to their respective constraint counterpart.

The resulting coordinates and labels are converted into \name{mec2} nodes and constraints as JSON-string.
This conversion makes it possible to convert the gathered information and make them processable for physic engines like \name{mec2}.
The creation of these JSON-strings is then tested in various environments.

A \name{Progressive Web App}(PWA) is created to nest this new functionality into an application.
For this the inference models are transformed to work in JavaScript to be able to make predictions from within other environments.

The \name{mec2} HTML element is embedded into this app to be able to apply detected nodes and constraints to a mechanical linkage.
This PWA is able to edit the mechanical linkage itself and implements the \name{mec2} API to offer all features the \name{mec2} HTML element does.

The PWA is then used as frontend for different implementations using WPF and WinUI.
These applications explore the possibility of linking the model to other programming languages, like C++, to improve performance.
Other ideas like communication between the PWA and a webserver to get better performance are also explored.


\section{Was erkannt werden soll}
\name{mec-2} modelliert Mechanismen durch
sogenannte \name{Nodes} und \name{Constraints}.
Beispielhaft für ein Viergelenk sähe das generierte JSON gemä{\ss} Abbildung \ref{fig:4bar} aus.

\begin{figure}
  \includegraphics[width=\textwidth]{images/4bar_json}
  \caption{Die JSON-Darstellung eines durch mec-2 erstellten Viergelenks mit dem entsprechend generierten Mechanismus.}
  \label{fig:4bar}
\end{figure}

Um diesen Mechanismus durch eine Handzeichnung zu erstellen, ist es erforderlich, einen Algorithmus zu trainieren, der in der Lage ist, mit einer entsprechenden Skizze als Input solchen JSON-Code als Output zu produzieren.

Hierfür mussten zunächst Trainingsdaten geschaffen werden, anhand derer ein solcher Algorithmus Merkmale erlernen kann, die eine Zuordnung der Eingangsbilder zu den entsprechenden Lösungen bilden kann.
Es wurden etwa 1200 Nodes, 1200 Base-Nodes und 1200 nicht zutreffende Bilder erstellt, welche vor dem Training durch Rotation und Spiegelung augmentiert wurden, um die Varianz der Trainingsdaten zu erhöhen.

Des Weiteren wurden von \name{mec-2} genutzte Symbole dem Trainingsset hinzugefügt, sodass der Algorithmus diese nicht als Nodes erkennt, damit Mechanismen erweitert werden können, ohne die bereits erkannten Elemente nochmal zu erkennen.

\begin{figure}
  \centering
    \begin{subfigure}[b]{0.4\textwidth}
        \includegraphics[width=\textwidth]{images/os.png}
        \caption{Nodes}
        \label{fig:os}
    \end{subfigure}
    \begin{subfigure}[b]{0.4\textwidth}
        \includegraphics[width=\textwidth]{images/xs.png}
        \caption{Base-Nodes}
        \label{fig:xs}
    \end{subfigure}
    \begin{subfigure}[b]{0.4\textwidth}
      \includegraphics[width=\textwidth]{images/rs.png}
      \caption{rotatorische Constraints}
      \label{fig:rs}
    \end{subfigure}
    \begin{subfigure}[b]{0.4\textwidth}
      \includegraphics[width=\textwidth]{images/ts.png}
      \caption{translatorische Constraints}
      \label{fig:ts}
    \end{subfigure}
    \caption{Beispiele für handgezeichnete Symbole welche zum Trainieren der Algorithmen genutzt werden.}
    \label{fig:example_symbols}
\end{figure}

Neben den Nodes wurden für die Constraints wieder 1200 rotatorische und 1200 translatorische Verbindungen gezeichnet, welche sich in ihrer Gestaltung an die Darstellung von Constraints an Abbildung \ref{fig:constraints_gtk} orientieren.
Für gebundene und freie Constraints gibt es noch keine Trainingsdaten, da diese für die momentan bestimmte Funktion zunächst irrelevant sind.

\begin{figure}
  \centering
  \includegraphics[width=0.8\textwidth]{images/gtk2019_tab1.png}
  \caption{Darstellung der Constraints in ihren unterschiedlichen Auslagemöglichkeiten. Auszug von Tabelle 1 des Beitrags von Prof. Gössner zum Sammelband des Getriebetechnischen Kolloquiums 2019 in Dortmund\cite{Goessner2019a}}.
  \label{fig:constraints_gtk}
\end{figure}

\section{Datenpipeline}
Als Konzeptnachweis ist ein neurales Netzwerk erstellt worden, welches die losen Nodes und Base-Nodes in den Abbildungen \ref{fig:os} und \ref{fig:xs} von Bildern unterscheiden kann, in denen eben keine Node enthalten ist. 

Dieses wird dann in ein \name{Fully Convolutional Neural Network}\cite{Long2014} umgewandelt, um diese Überprüfung auf einem Bild zusätzlich lokalisieren zu können.

Ein weiteres neurales Netzwerk wird nun erstellt, um die \name{Constraints} zu erkennen.
Hierfür wurden die Symbole, wie sie in den Abbildungen \ref{fig:rs} und \ref{fig:ts} zu sehen sind, benutzt, um Trainingsdaten zu generieren, die wiederum genutzt werden können, um das neurale Netzwerk zu trainieren.
Hierfür wurden die vorher genutzten Nodes zufällig auf einem Bild verteilt und daraufhin zufällig durch Constraints verbunden.

\begin{figure}
  \centering
    \begin{subfigure}[b]{0.3\textwidth}
        \includegraphics[width=\textwidth]{images/pre_crop.png}
        \caption{}
        \label{fig:pre_crop}
    \end{subfigure}
    \begin{subfigure}[b]{0.3\textwidth}
      \includegraphics[width=\textwidth]{images/crops.png}
      \caption{}
      \label{fig:crop}
    \end{subfigure}
    \caption{Aus dem links zufällig generiertem Bild wurden die Bilder rechts generiert. Anzumerken ist, dass nur die Bilder der ersten Reihe als korrekt bezeichnet werden.}
    \label{fig:constraint_data}
\end{figure}

Die daraus entstandenen Schnitte werden verformt, um einheitliche Maße zu erhalten.
Des Weiteren werden die Bilder so gespiegelt, dass die erste Node stets oben links im Bild ist.
So sind Bilder nur als korrekt anzusehen, wenn die Verbindung von oben links nach unten rechts dargestellt wird.
Das ist notwendig, um neben der Kategorisierung auch die Richtung der Verbindung zu bestimmen.

Aufbauend auf diesen Algorithmen kann nun eine Datenpipeline programmiert werden, welche anhand eines Bildes zunächst die Nodes erkennt und daraufhin die Bildausschnitte generiert, welche dahingehend untersucht werden können, ob zwischen zwei Nodes eine Constraints existiert.

Die daraus gewonnene Information kann dann in das von \name{mec-2} genutzte JSON Format überführt und der Mechanismus kann dargestellt werden.
Ein vollständig definierter Mechanismus (alle Gelenke und Glieder, die Anzahl der Freiheitsgrade etc. seien dadurch bekannt) lässt sich so direkt der entsprechenden kinematischen Kette zuordnen.

\section{Training}
Während die Anwendung des Programms vollständig in JavaScript erfolgt und so leicht in eine Webapplikation einzubetten ist, geschieht das Training der neuralen Netzwerke aus Gründen der Performanz in Python.
Mithilfe von CUDA \cite{nvidia2019} kann so direkt mit den Grafikprozessoren des PCs gearbeitet werden\footnote{Diese Zugriffe werden für den Entwickler durch die genutzten Bibliotheken \name{Tensorflow} und \name{Keras} abstrahiert.}.

Das Erlernen von Merkmalen geschieht durch das Minimieren einer Verlustfunktion, welche die Differenz zwischen den vom Modell bestimmten Ergebnis und dem tatsächlichen Ergebnis misst \cite[S.710]{StuartRussell2018}.
Die Ableitung dieser Verlustfunktion nach dem Output des Modells wird dann auf das neuronale Netzwerk angewandt, um so mutma{\ss}lich bei der nächsten Vorhersage einen niedrigeren Wert als Verlust zu bekommen \cite[S.719]{StuartRussell2018}.

Dieser Prozess wiederholt sich hinreichend oft, bis die Vorhersagen eine angestrebte Genauigkeit erreichen.

Ein neurales Netzwerk enthält initial zufällig zugewiesene Werte für die einzelnen Zellen, sodass bei drei Kategorien mit einer Genauigkeit von etwa 33\% auszugehen ist.
Durch gewählte Trainingsparameter kann die Genauigkeit des Modells erhöht werden.
So erreicht das neuronale Netzwerk zuständig für die Node-Erkennung eine Genauigkeit von 99,75\% bei Daten, welche vor dem Training separiert wurden.
Das Modell für die Erkennung von Constraints erreicht hier eine Genauigkeit von 97,68\%.

Es ist anzumerken, dass die Genauigkeit durch Daten ermittelt wird, welche vom Modell nie gesehen wurden, entsprechend also kein Auswendiglernen der Zuordnungen \cite[S.705]{StuartRussell2018} für die Genauigkeit verantwortlich sein kann.
Sie sind allerdings nur in der Lage, jene Daten korrekt zuzuordnen, welche dem Format der Trainingsdaten entsprechen.

Die Trainingsdaten wurden bewusst generiert, um einen Kompromiss zwischen ausreichendem Material für das Erlernen der notwendigen Merkmale in den Bildern zu finden, allerdings ohne tausende Mechanismen zeichnen und notwendigerweise beschriften zu müssen.

Die aktuell genutzten Modelle für das Erkennen von Nodes und Constraints verwenden zum Training ausschlie{\ss}lich wei{\ss}e Symbole auf schwarzem Hintergrund\footnote{Die Abbildungen in dieser Ausarbeitung sind invertiert, um die Sichtbarkeit zu verbessern.}.
Merkmale können ausschlie{\ss}lich in dieser Gestalt erkannt werden. Beispielsweise würde ein schwarzer Kreis auf wei{\ss}em Hintergrund nicht als Node erkannt werden, da der Algorithmus diesem die zur korrekten Zuordnung erforderlichen Merkmale nicht zuordnen kann.

\section{Beispiele}
Diese Beispiele wurden in einer dafür konzipierten Erweiterung für das \name{mec2} HTML Element gezeichnet.
Komplexere Skizzen von Mechanismen in einem Durchgang zu erkennen, ist mit den aktuellen Modellen noch mühselig.
Die Modelle müssten vorab für den genutzten Anwendungsbereich trainiert werden, um die korrekte Funktionsweise zu gewährleisten.

\begin{figure}[H]
    \centering
    \begin{subfigure}[b]{0.3\textwidth}
        \includegraphics[width=\textwidth]{images/4bar_sketch.png}
        \caption{}
        \label{fig:4bar_sketch}
    \end{subfigure}
    \begin{subfigure}[b]{0.3\textwidth}
        \includegraphics[width=\textwidth]{images/4bar_prediction.png}
        \caption{}
        \label{fig:4bar_prediction}
    \end{subfigure}
    \label{fig:4bar_example}
    \caption{Ein Viergelenk}
\end{figure}

\begin{figure}[H]
    \centering
    \begin{subfigure}[b]{0.3\textwidth}
        \includegraphics[width=\textwidth]{images/pump_sketch.png}
        \caption{}
        \label{fig:pump_sketch}
    \end{subfigure}
    \begin{subfigure}[b]{0.3\textwidth}
        \includegraphics[width=\textwidth]{images/pump_prediction.png}
        \caption{}
        \label{fig:pump_prediction}
    \end{subfigure}
    \label{fig:pump_example}
    \caption{Mechanismus mit translatorischem Glied}
\end{figure}

\printbibliography

\end{document} 