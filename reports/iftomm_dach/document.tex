% !TeX encoding=utf8
% !TeX program = pdflatex
% !BIB = biber

\listfiles % listet alle geladenen Pakete im .log file
% sollte ihr .tech-file nicht kompilieren, so vergleichen Sie bitte Ihre Liste
% mit PaketListe.txt

\documentclass[paper=a5,
      fontsize=10pt,
      parskip=half-, %
      ngerman   % Neue Rechtschreibung, d.\,h. (Silbentrennung)
]{scrartcl}

\usepackage{packages} % Präambel laden
% NOTE selbstgemacht...
\newcommand\name[1]{\texttt{#1}}
% NOTE selbstgemacht ende

% Laden der .bib Datei
\addbibresource{document.bib}

% TITEL des Beitrags % \LARGE zur Erzeugung korrekter Überschriftengröße
\title{\LARGE{Automatische Klassifizierung handgezeichneter Mechanismen durch maschinelles Lernen
}}

% AUTOREN (Vorname, Nachname)
\author{Stefan Gössner*, Kai Lawrence**}

% DATUM ausblenden
\date{}

% dieses Feld wird benutzt um die Kontaktadresse darzustellen
% bitte nutzen sie ein hochgestellten Stern (*) um bei mehreren Instituten zu
%unterscheiden
\publishers{* FH Dortmund, Professur für Dynamik, Mechnanismentechnik und Webtechnologien\\
	stefan.gössner@fh-dortmund.de
\\ \vspace{\baselineskip}
% Bei mehreren Instituten nutzen Sie bitte mehrere hochgestellte Sterne, um die
%Autoren zuzuordnen
** FH Dortmund, Maschinenbaustudent\\
	kaihenning.lawrence002@stud.fh-dortmund.de}


\begin{document}

% Wahl der Sprache des Dokuments (ngerman oder englisch)
\selectlanguage{ngerman}

\maketitle % erzeuge Titel

\section*{Kurzfassung}
Für eine schnelle, digitale Skizze zweidimensionaler Mechanismen erlauben neue Technologien im Vergleich zu aktuellen Methoden komfortablere Formen der Benutzerinteraktion.
Es gibt zwar bereits Anwendungen zur schnellen und intuitiven Erstellung von Mechanismen, aber für kurze Betrachtungen ist der Einsatz solcher Editoren in vielen Fällen nicht der Mühe wert.

Um auch ad-hoc eine Möglichkeit zu haben, Mechanismen interaktiv oder animierbar zu betrachten, wird nun ein Programm entwickelt, welches es ermöglichen soll, Handzeichnungen zu scannen und den respektiven Mechanismus digital zu ermitteln.

\section*{Abstract}
For fast, digital sketching of two-dimensional mechanisms, new technologies allow more comfortable forms of user interaction compared to current methods.
Although there are already applications for the fast and intuitive creation of mechanisms, for brief observations the use of such editors is in many cases not worth the effort.

In order to be able to view mechanisms interactively or animatedly on an ad-hoc basis, a program is now being developed which will allow hand drawings to be scanned and the respective mechanism to be determined digitally.


\printbibliography

\end{document} 