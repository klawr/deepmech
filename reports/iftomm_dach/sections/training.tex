\section{Training}
Während die Anwendung des Programms vollständig in JavaScript erfolgt und so leicht in eine Webapplikation einzubetten ist, geschieht das Training der neuralen Netzwerke aus Gründen der Performanz in Python.
Mithilfe von CUDA \cite{nvidia2019} kann so direkt mit den Grafikprozessoren des PCs gearbeitet werden\footnote{Diese Zugriffe werden für den Entwickler durch die genutzten Bibliotheken \name{Tensorflow} und \name{Keras} abstrahiert.}.

Das Erlernen von Merkmalen geschieht durch das Minimieren einer Verlustfunktion, welche die Differenz zwischen den vom Modell bestimmten Ergebnis und dem tatsächlichen Ergebnis misst \cite[S.710]{StuartRussell2018}.
Die Ableitung dieser Verlustfunktion nach dem Output des Modells wird dann auf das neuronale Netzwerk angewandt, um so mutma{\ss}lich bei der nächsten Vorhersage einen niedrigeren Wert als Verlust zu bekommen \cite[S.719]{StuartRussell2018}.

Dieser Prozess wiederholt sich hinreichend oft, bis die Vorhersagen eine angestrebte Genauigkeit erreichen.

Ein neurales Netzwerk enthält initial zufällig zugewiesene Werte für die einzelnen Zellen, sodass bei drei Kategorien mit einer Genauigkeit von etwa 33\% auszugehen ist.
Durch gewählte Trainingsparameter kann die Genauigkeit des Modells erhöht werden.
So erreicht das neuronale Netzwerk zuständig für die Node-Erkennung eine Genauigkeit von 99,75\% bei Daten, welche vor dem Training separiert wurden.
Das Modell für die Erkennung von Constraints erreicht hier eine Genauigkeit von 97,68\%.

Es ist anzumerken, dass die Genauigkeit durch Daten ermittelt wird, welche vom Modell nie gesehen wurden, entsprechend also kein Auswendiglernen der Zuordnungen \cite[S.705]{StuartRussell2018} für die Genauigkeit verantwortlich sein kann.
Sie sind allerdings nur in der Lage, jene Daten korrekt zuzuordnen, welche dem Format der Trainingsdaten entsprechen.

Die Trainingsdaten wurden bewusst generiert, um einen Kompromiss zwischen ausreichendem Material für das Erlernen der notwendigen Merkmale in den Bildern zu finden, allerdings ohne tausende Mechanismen zeichnen und notwendigerweise beschriften zu müssen.

Die aktuell genutzten Modelle für das Erkennen von Nodes und Constraints verwenden zum Training ausschlie{\ss}lich wei{\ss}e Symbole auf schwarzem Hintergrund\footnote{Die Abbildungen in dieser Ausarbeitung sind invertiert, um die Sichtbarkeit zu verbessern.}.
Merkmale können ausschlie{\ss}lich in dieser Gestalt erkannt werden. Beispielsweise würde ein schwarzer Kreis auf wei{\ss}em Hintergrund nicht als Node erkannt werden, da der Algorithmus diesem die zur korrekten Zuordnung erforderlichen Merkmale nicht zuordnen kann.
