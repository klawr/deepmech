\section*{Kurzfassung}
Anwendungen für Mechanismentechnik leiden immer unter dem Problem, dass für eine schnelle Skizze eines Mechanismus die entsprechenden Benutzeroberflächen der bestehenden Mechanismeneditoren mit Aufwand verbunden sind. Es gibt zwar Anwendungen zur schnellen und intuitiven Erstellung von Mechanismen, aber für kurze Betrachtungen ist der Einsatz eines solchen Editors in vielen Fällen nicht der Mühe wert.

Um auch ad-hoc eine Möglichkeit zu haben, Mechanismen interaktiv oder animiert zu betrachten, wird nun ein Programm entwickelt, welches es ermöglichen soll, Handzeichnungen zu scannen und den respektiven Mechanismus digital zu ermitteln.

\section*{Abstract}
Applications for mechanism engineering always suffer from the problem that for a quick sketch of a mechanism the corresponding user interfaces of existing mechanism editors are connected with effort.
Applications for a quick and intuitive sketching of mechanisms exist, but for short considerations using a computer is in many cases not worth the effort.

In order to be able to view mechanisms interactively or animatedly on an ad-hoc basis, a program is now being developed which will allow hand drawings to be scanned and the respective mechanism to be determined digitally.
