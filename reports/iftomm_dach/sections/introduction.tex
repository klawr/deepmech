\section{Einführung}
Die Simulation, Analyse und Synthese von Mechanismen mittels Basiswebtechnologien stehen im Fokus der Arbeitsgruppe \name{Mechanismentechnik} und \name{Webtechnologien} im Fachbereich Maschinenbau der Fachhochschule Dortmund.
Im Rahmen dieser Tätigkeiten sind die in JavaScript implementierten Bibliotheken \name{mec2.js} \cite{Goessner2019, Goessner2019a, Goessner2018}, einer Bibliothek zur Modellierung ebener Mechanismen auf der Grundlage einer partikelzentrierten Physik-Engine, und \name{g2.js} \cite{Goessner2019b} als minimalistische Grafikbibliothek entstanden. 

Anwendung finden diese bereits in einem speziellen \name{<mec-2>} HTML-Element und dem webbasierten Mechanismeneditor \name{mecEdit} \cite{Uhlig2019, Uhlig2019a}, der bereits aktiv in der Lehre genutzt wird.

Webanwendungen stehen stets unter dem Vorsatz, eine möglichst reibungsfreie Nutzerinteraktion zu gewährleisten.
\name{mecEdit} bietet hierfür dem Nutzer die Möglichkeit, per möglichst einfacher Eingabe der entsprechenden Elemente des Mechanismus diesen interaktiv und animierbar auf den Bildschirm zu bringen.

Für eine schnelle Skizze ist bislang der schnellste Weg die Handzeichnung auf dem Papier, welche jedoch intrinsisch weder interaktiv noch animierbar ist.

Im Folgendem wird das Ziel der Zusammenführung der Vorzüge der analogen und digitalen Herangehensweise beschrieben.
Der Prozess soll handskizzierte Mechanismen durch Mustererkennung kinematischen Modellen zuordnen. Bilder werden dafür in ein Programm eingeladen, durch Mustererkennung analysiert und das \name{mec2} Modell ermittelt.

Hierfür wurden im Sinne des maschinellen Lernens mehrere statistische Modelle trainiert, welche in der Lage sind, die durch \name{mec2} beschriebenen \name{Nodes} und \name{Constraints} zu erkennen und daraus einen entsprechenden Mechanismus abzuleiten.

Für das Training des Modells wurden die Bibliotheken \name{Tensorflow} \cite{Google2019} und \name{Keras} \cite{Chollet2019, Chollet2017} genutzt.
