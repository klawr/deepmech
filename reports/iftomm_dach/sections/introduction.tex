\section{Einführung}
Die Simulation, Analyse und Synthese von Mechanismen mittels Basiswebtechnologien stehen im Fokus der Arbeitsgruppe \name{Mechanismentechnik} und \name{Webtechnologien} im Fachbereich Maschinenbau der Fachhochschule Dortmund.
Im Rahmen dieser Tätigkeiten sind die in JavaScript implementierten Bibliotheken \name{mec2.js}\cite{Goessner2019}\cite{Goessner2019a}\cite{Goessner2018}, einer Bibliothek zur Modellierung ebener Mechanismen auf der Grundlage einer partikelzentrierten Physik-Engine, und \name{g2.js}\cite{Goessner2019b} als minimalistische Grafikbibliothek entstanden. 

Anwendung finden diese bereits in einem speziellen \name{<mec-2>} HTML-Element und dem webbasierten Mechanismeneditor \name{mecEdit}\cite{Uhlig2019}\cite{Uhlig2019a}, der bereits aktiv in der Lehre genutzt wird.

Webanwendungen stehen stets unter dem Vorsatz, eine möglichst reibungsfreie Nutzerinteraktion zu gewährleisten.
\name{mecEdit} bietet hierfür dem Nutzer die Möglichkeit, per möglichst einfacher Eingabe der entsprechenden Elemente des Mechanismus diesen interaktiv und animiert auf den Bildschirm zu bringen.

Für eine schnelle Skizzierung ist jedoch bislang der schnellste Weg die Handzeichnung auf dem Papier, welche jedoch intrinsisch weder interaktiv noch animiert ist.

Die Vorzüge der analogen und digitalen Herangehensweise zusammenzuführen stellt nun eine eine weitere Schnittstelle in der Entwicklung dar.
Sie soll handskizzierte Mechanismen durch Mustererkennung kinematischen Modellen zuordnen, indem entsprechend in das Programm eingeladene Bilder analysiert und das \name{mec-2} Modell ermittelt werden kann.

Hierfür sind mehrere statistische Modelle trainiert worden, welche in der Lage sind, sowohl die durch \name{mec2} beschriebenen \name{Nodes} und \name{Constraints} zu erkennen und daraus einen entsprechenden Mechanismus abzuleiten.

Für das Training des Modells wurden die Bibliotheken \name{Tensorflow}\cite{Google2019} und \name{Keras}\cite{Chollet2019}\cite{Chollet2017} genutzt.
