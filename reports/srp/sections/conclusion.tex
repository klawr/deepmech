\section{Conclusion}

After examining the possibilities of machine learning and their impact on the different branches of technology, the idea to extend this into the realm of kinematic seems to be a reasonable one.
This project was used to learn the basics needed to apply machine learning from scratch, beginning with a simple statistical model which is able to classify handwritten mechanical symbols.

The data suggests that a sufficient algorithm has been found, which has an accuracy of over 99\% on the test data.
After removing the training configuration it has a size of 76kb as \name{Keras} model.
Converting the model to a format which is readable by \name{tensorflow.js}\footnote{\url{https://www.tensorflow.org/js/}} and assembling it into a readable \code{IOHandler}\footnote{The loader.js can be found at \url{https://aka.klawr.de/srp\#15}} does not increase size significantly (81kb).

Applications to test the models have been provided in the repository:

A Python-script to draw images into the environment used to create data, but used to predict the drawn symbol can be found in the \name{code} directory as \name{symbol\_classifier\_test.py}\footnote{\url{https://aka.klawr.de/srp\#16}}.

A web page which can be loaded directly inside a web-browser\footnote{The test works on Firefox and Google Chrome. Microsoft Edge does not work yet, but this is probably subject to change when Microsoft switches to a Chromium build. }.
The file can be found in the same directory as the Python-script as \name{symbol\_classifier\_test.html}\footnote{\url{https://aka.klawr.de/srp\#17}}.
This test is close to the intended purpose by uploading images to the application and then providing the user with automatic predictions.

Subsequent projects will follow to extend the functionality to detect the position of detected symbols in arbitrary images and possible connections in the form of constraints between them.
After full mechanism representations can be deduced they can be imported into existing web applications like \name{mec2} \cite{Goessner2019} \cite{Goessner2019a} \cite{Goessner2019b} or \name{mecEdit} \cite{Uhlig2019} \cite{Uhlig2019a}.
