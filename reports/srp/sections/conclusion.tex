\section{Conclusion}

After investigating the possibilities of machine learning and its impact on the different branches of technology, the idea of extending this to the field of kinematics seems reasonable.
In this project the basics for the application of machine learning were learned from scratch, starting with a simple statistical model capable of classifying handwritten mechanical symbols.

The data suggest that a sufficient algorithm has been found that has an accuracy of over 99\% over the test data.
After removing the training configuration, it has a size of 76kb as \name{Keras} model.
Converting the model into a format readable by \name{tensorflow.js}\footnote{\url{https://www.tensorflow.org/js/}} and assembling it into a readable \code{IOHandler}\footnote{The loader.js can be found at \url{https://aka.klawr.de/srp\#15}} does not significantly increase the size (81kb).

Applications for testing the models were made available in the repository:

A Python script for testing purposes to draw images and the model predicting the drawn symbol, is located in the \name{code} directory as \name{symbol\_classifier\_test.py}\footnote{\url{https://aka.klawr.de/srp\#16}}.

A web page that can be loaded directly into a web browser\footnote{The test works on Firefox and Google Chrome. Microsoft Edge does not yet work, but this will probably change when Microsoft moves to a Chromium-based build. }.
The file is located in the same directory as the Python script as \name{symbol\_classifier\_test.html}\footnote{\url{https://aka.klawr.de/srp\#17}}.
This test comes close to its intended purpose by uploading images into the application and then providing automatic predictions to the user.

Later projects will follow to provide functionality for detecting the position of recognized symbols in a given image.
Furthermore, possible connections of the symbols to each other in the form of "constraints" will be recognized.
After complete mechanism representations can be derived, they can be imported into existing web applications like \name{mec2} \cite{Goessner2019, Goessner2019a, Goessner2019b} or \name{mecEdit}. \cite{Uhlig2019, Uhlig2019a}.