\section{Data Generation}

Having introduced all concepts used in this project, I want to get to the creation of
\name{deepmech}. Before talking about how to create appropriate models for the detection of
mechanical symbols, it is necessary to gather appropriate data for that task.

Since the goal is to read handwritten data, it is obvious that this is accomplished by
creating the data myself. For this a Python script is written which allows to rapidly
draw simple symbols and safe them on the disk\footnote{ Python is used because of easy integration
to the file system, which allows for easy saving images without the need for intermediate
steps. }.
There were several requests for the script to be fulfilled. The images had to be drawn on a
canvas which is set to 

The code can be reviewed at \url{https://klawr.github.io/deepmech/src/data/data_generation.py}
and is also in Appendix~\ref{appendix:a}.

For the data generation a few things had to be considered:

\begin{enumerate}
\item In which state are the images going to be read?
\item Which symbols are to be tested first?
\item How should the images be tested?
\end{enumerate}
