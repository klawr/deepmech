\section{Introduction}

Artificial Intelligence (AI) is a term which was coined in TODO \footnote{ ... } and promised to
outsmart humans multiple times in its former history.
Not delivering these promises more silent phases which are now known as
AI-Winter \footnote { TODO } emerged.
In recent times however research and application of these techniques are again attracting more
attention which is shown by figure TODO .

This is due to some factors which are becoming prevalent in current times. On one hand it is the
shear amount of computing power which is now standard in computers. Even home computers are
nowadays capable of training simple to even not so simple TODO models which are capable of
impressive tasks \footnote { TODO }. Modern super computers are capable of very complex tasks like
TODO \footnote { TODO }.

Another catalysing TODO factor is the amount of data which is available due to the internet.
To train a reasonable model for an artificial network for a specific task suitable data is needed
make correct predictions given new data (generalization).
Since data is available in abundancy this is no concern in modern times.
Therefore AI is getting more attention in fields of research and more and more practical fields
are considering this technique as solution for various problems.

Even though artificial intelligence is used to describe the broader spectrum of appliances,
this project focusses on the machine learning part of this interesting TODO.
In this project I want to introduce machine learning to the field of kinematics for two
dimensional sketching and prototyping with \name{deepmech}.

\name{deepmech} takes images as input and outputs the type and location of handwritten
mechanical symbols inside of it.
The models created by the presented approach are small and mostly language agnostic,
which is provided by the usage of the popular Keras\footnote{ TODO } library on top
of TensorFlow\footnote{ TODO }.

In this paper, simple demonstrations are done using JavaScript, connecting to a new emerging
field in kinematics using web technologies\footnote{ TODO }.
For training of the actual model the Python implementation of TensorFlow is used,
because of the better utilization of the GPU using backends like CUDA and CUDNN\footnote{ TODO }
which results in faster training. Thanks to mostly language agnostic model descriptions,
this allows for seamless transitions between different approaches.

Before showing how deepmech was developed, the basics of machine learning shall be covered,
showing some simple examples as introduction and by doing so introduce common terminology used in
this paper.
